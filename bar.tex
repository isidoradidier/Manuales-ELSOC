% Options for packages loaded elsewhere
\PassOptionsToPackage{unicode}{hyperref}
\PassOptionsToPackage{hyphens}{url}
%
\documentclass[
]{book}
\usepackage{amsmath,amssymb}
\usepackage{lmodern}
\usepackage{ifxetex,ifluatex}
\ifnum 0\ifxetex 1\fi\ifluatex 1\fi=0 % if pdftex
  \usepackage[T1]{fontenc}
  \usepackage[utf8]{inputenc}
  \usepackage{textcomp} % provide euro and other symbols
\else % if luatex or xetex
  \usepackage{unicode-math}
  \defaultfontfeatures{Scale=MatchLowercase}
  \defaultfontfeatures[\rmfamily]{Ligatures=TeX,Scale=1}
\fi
% Use upquote if available, for straight quotes in verbatim environments
\IfFileExists{upquote.sty}{\usepackage{upquote}}{}
\IfFileExists{microtype.sty}{% use microtype if available
  \usepackage[]{microtype}
  \UseMicrotypeSet[protrusion]{basicmath} % disable protrusion for tt fonts
}{}
\makeatletter
\@ifundefined{KOMAClassName}{% if non-KOMA class
  \IfFileExists{parskip.sty}{%
    \usepackage{parskip}
  }{% else
    \setlength{\parindent}{0pt}
    \setlength{\parskip}{6pt plus 2pt minus 1pt}}
}{% if KOMA class
  \KOMAoptions{parskip=half}}
\makeatother
\usepackage{xcolor}
\IfFileExists{xurl.sty}{\usepackage{xurl}}{} % add URL line breaks if available
\IfFileExists{bookmark.sty}{\usepackage{bookmark}}{\usepackage{hyperref}}
\hypersetup{
  pdftitle={ESTUDIO LONGITUDINAL SOCIAL DE CHILE (ELSOC 2016)},
  hidelinks,
  pdfcreator={LaTeX via pandoc}}
\urlstyle{same} % disable monospaced font for URLs
\usepackage{graphicx}
\makeatletter
\def\maxwidth{\ifdim\Gin@nat@width>\linewidth\linewidth\else\Gin@nat@width\fi}
\def\maxheight{\ifdim\Gin@nat@height>\textheight\textheight\else\Gin@nat@height\fi}
\makeatother
% Scale images if necessary, so that they will not overflow the page
% margins by default, and it is still possible to overwrite the defaults
% using explicit options in \includegraphics[width, height, ...]{}
\setkeys{Gin}{width=\maxwidth,height=\maxheight,keepaspectratio}
% Set default figure placement to htbp
\makeatletter
\def\fps@figure{htbp}
\makeatother
\setlength{\emergencystretch}{3em} % prevent overfull lines
\providecommand{\tightlist}{%
  \setlength{\itemsep}{0pt}\setlength{\parskip}{0pt}}
\setcounter{secnumdepth}{-\maxdimen} % remove section numbering
\ifluatex
  \usepackage{selnolig}  % disable illegal ligatures
\fi

\title{ESTUDIO LONGITUDINAL SOCIAL DE CHILE (ELSOC 2016)}
\usepackage{etoolbox}
\makeatletter
\providecommand{\subtitle}[1]{% add subtitle to \maketitle
  \apptocmd{\@title}{\par {\large #1 \par}}{}{}
}
\makeatother
\subtitle{MANUAL DE USUARIO DE BASE DE DATOS CORTE TRANSVERSAL}
\author{true \and true \and true \and true \and true}
\date{2021-06-08}

\begin{document}
\frontmatter
\maketitle

\mainmatter
\hypertarget{present}{%
\chapter{Presentación}\label{present}}

El Estudio Longitudinal Social de Chile (ELSOC) es una encuesta
desarrollada para analizar intertemporalmente la evolución del conflicto
y cohesión en la sociedad chilena, basándose en modelos conceptuales
descritos en la literatura nacional e internacional. De este modo, se
orienta a examinar los principales antecedentes, factores moderadores y
mediadores, así como las principales consecuencias asociadas al
desarrollo de distintas formas de conflicto y cohesión social en Chile.
Por lo tanto, su objetivo fundamental es constituirse en un insumo
empírico para la comprensión de las creencias, actitudes y percepciones
de los chilenos hacia las distintas dimensiones de la convivencia y el
conflicto, y como éstas cambian a lo largo del tiempo.

Esta encuesta fue diseñada por investigadores pertenecientes al Centro
de Estudios de Conflicto y Cohesión Social (COES). El Centro está
patrocinado por la Universidad de Chile y la Pontificia Universidad
Católica de Chile y cuenta como instituciones asociadas a la Universidad
Diego Portales y la Universidad Adolfo Ibáñez. Si desea obtener más
información sobre el Centro, visite nuestra
\href{http://www.coes.cl/}{página web}. COES es una iniciativa que desde
2013 cuenta con el financiamiento\footnote{Proyecto
  CONICYT/FONDAP/15130009} del Fondo de Financiamiento de Centros de
Investigación en Áreas Prioritarias (FONDAP) de la Comisión Nacional de
Investigación Científica y Tecnológica (CONICYT), organismo dependiente
del Ministerio de Educación de Chile. El levantamiento de datos de ELSOC
(primera ola de medición 2016) estuvo a cargo del CentroMicroDatos (CMD)
de la Universidad de Chile.

El presente documento contiene información relevante para los usuarios
de la base de datos de la primera ola del estudio panel, correspondiente
al año 2016, enfatizando su dimensión de corte transversal. El Manual se
estructura en torno a secciones temáticas. La siguiente sección describe
brevemente el diseño del instrumento, lo cual es complementado con una
ficha técnica del estudio. La tercera sección reseña el diseño muestral
general del panel como también los detalles específicos de la primera
ola del estudio. En el siguiente apartado se resumen los principales
aspectos del trabajo de campo y ejecución del estudio. El quinto
apartado corresponde al registro de versiones de la base de datos y
detalla el protocolo de uso de ésta. Por último, se incluye un apartado
con orientaciones básicas para el análisis y el libro de códigos de las
variables incluidas en la base de datos.

\backmatter
\end{document}
